

%----------------------------------------------------------------------------------------
%	ADDRESSEE SECTION
%----------------------------------------------------------------------------------------
\documentclass[10pt]{letter} % 10pt font size default, 11pt and 12pt are also possible

\usepackage{geometry} % Required for adjusting page dimensions

%\longindentation=0pt % Un-commenting this line will push the closing "Sincerely," to the left of the page

\geometry{
	paper=a4paper, % Change to letterpaper for US letter
	top=2cm, % Top margin
	bottom=1.5cm, % Bottom margin
	left=2.5cm, % Left margin
	right=2.5cm, % Right margin
	%showframe, % Uncomment to show how the type block is set on the page
}

\usepackage[T1]{fontenc} % Output font encoding for international characters
\usepackage[utf8]{inputenc} % Required for inputting international characters
\usepackage{color}
\usepackage{stix} % Use the Stix font by default

\usepackage{microtype} % Improve justification
\usepackage{dashrule}

%----------------------------------------------------------------------------------------
%	YOUR NAME & ADDRESS SECTION
%----------------------------------------------------------------------------------------

%\signature{John Smith} % Your name for the signature at the bottom

%\address{123 Broadway \\ City, State 12345 \\ (000) 111-1111} % Your address and phone number

%----------------------------------------------------------------------------------------
\newcommand{\mdf}[1]{\textcolor[rgb]{1.00,0.00,1.00}{#1}}
\begin{document}
	\textbf{Dear Editor,}\\
	Thank you for handling our manuscript. Our submission paper1027 entitled ``Preventing Self-intersection with Cycle Regularization in Mesh
	Reconstruction Networks'' is reviewed in GMP2019 and recommended to your journal for a fast-track submission process. We find the reviewers’ comments to be very valuable and helpful in improving our presentation, as well as important for guiding significantly to our research. We have read the comments carefully and the manuscript have been rechecked. Our modifications are provided in the revised manuscript according to these comments and suggestions. All the modifications are \mdf{highlighted in magenta} in the revised manuscript by the macro $\backslash$mdf in the source file. Listed below are our point-by-point responses to the reviewer comments.\\
	With all best wishes,\\
	Sincerely,
	
	\hdashrule{\linewidth}{1pt}{1mm}
	Chair:
	
	Q: \emph{The revision should focus on the case of models of non-zero genus.}
	
	A: As a response to chair's directives, we have added extra experiments in Sec 4.2 of our revised paper to try our proposed cycle regularization on case of models of genus-1. We discussed the effect of our cycle regularization in such cases. It is also our opinion that how to enable the neural network to learn to predict surface with arbitrary genus remains a challenging problem and is beyond the scope of this paper. 
	
	\hdashrule{\linewidth}{1pt}{1mm}
	%----------------------------------------------------------------------------------------
	Reviewer 21833:\\
	Q: \emph{I recommend to state the 2D image input directly in the title.}
	
	A:We have changed our title to ``Preventing Self-intersection with Cycle Regularization in Single View Mesh Reconstruction Networks''
	
	Q: \emph{The paper should also better distinguish between local an global self-intersections.}
	
	A:We have rewritten our discussion about self-intersections in introduction to better distinguish the local and global interference, in which we have included more relevant references you recommended.
	
	Q: \emph{The structure of the paper is quite unfortunate. For example Section 3.1. refers to Definition 1 that appears later; this is very reader-unfriendly.}
	
	A:We have adjusted our manuscript to avoid such case as much as we can.
	
	Q:\emph{Moreover, the paper speaks about injectivity of $f$ and self-overlapped points like these were two different phenomena.}
	
	A:Injectivity is a property of the regressed function $f$. Self-overlapped 3D points are observed defects on surface meshes. They maybe naturally related in the study of parameterization but not in general. For example, there is a previous shape generation network PSGN( Fan et al. (2017) ) who regress orderless point set from 2D image input. You can hardly relate the self-overlapped points to injectivity of the regressed function $f$, because in such manner the pre-image of the fucntion $f$ is in domain of 2D image and the image of the function $f$ is in domain of Nx3 matrix. Only in latest works as AtlasNet and Pixel2Mesh, it is possible to relate self-overlapped 3D points to $f$'s injectivity. Such benefit comes from learning mapping of points instead of directly regressing point coordinates of the point set from image input. This is the foundation that makes our idea possible. Therefore, it is important for us to distinguish the concepts of function $f$'s injectivity and the self-overlapped 3D points in mesh.
	
	Q:\emph{Another rather confusing part is Def. 1 itself. Why do you formulate (1) and (2)? These are obviously equivalent and there is no reason to state them both.}
	
	A:We only keep statement (2) in our revised paper.
	
	Q:\emph{Fig.8 is very small and the zoom-in is barely visible.  Some other figures should be provided in higher resolution, e.g. the *Input* row in Fig.6 is really bad.} 
	
	A:For input images, the images from the benchmark dataset are all low resolution images(224x224). This is now the standard setting with neural networks. It is difficult for us to provide high resolution version of them.
	For output meshes, we have adjusted the images in our revised paper. We also added the output meshes for the paper into supplemental material for your review.
	
	Q:\emph{Looking at fig. 4, One wonders how many iterations are needed to receive results with desired chamfer distance. How do you set the number of iterations in your algorithm; what is the termination criterion?}
	
	A:We have rewritten Sec 4.1 to elaborate on more experiment details. In original manuscript this experiment was only meant to provide more visible insight into the effect of our proposed cycle regularization during optimization. Now we have put more time into it to make it a more serious experiment. 
	
	Q:\emph{What about objects with other genus than zero?}
	
	A:Solving this problem requires enabling neural network to learn to predict surfaces of arbitrary genus, which remains a challenge and beyond the scope of this paper. However, we do small exploration into the cases of genus-1 by manually selecting torus as source surface in our revised paper. Please see our Sec 4.2 for more details.
	
	\hdashrule{\linewidth}{1pt}{1mm}
	%----------------------------------------------------------------------------------------
	Reviewer 21924:\\
	Thank you for your approval. 
	
	We have added your recommendation of references as related work in our revised manuscript. Getting rid of the drawback you have pointed out is beyond our reach for now. We will keep working on the problem.
	  
	
	\hdashrule{\linewidth}{1pt}{1mm}
	%----------------------------------------------------------------------------------------
	Reviewer 21930:\\
	Thank you for your approval.
	
	Q: \emph{In Proposition 1, the free variable x should be bound by a quantifier.
		\\The first sentence of section 3.3 is describing a version of the Universal Approximation Theorem and would do well to refer to this.}
	
	A: We have fixed these two issues in our revised paper.
	
	Q:\emph{ A rather stringent limitation of the method is that it only applies to models with zero genus.}
	
	A: Lifting such limitation requires enabling neural network to learn to predict surfaces of arbitrary genus, which is still beyond our reach for now. However, we add some small exploration and discussion into the cases of genus-1. Please see Sec 4.2 for details. 

	
\end{document}