

%----------------------------------------------------------------------------------------
%	ADDRESSEE SECTION
%----------------------------------------------------------------------------------------
\documentclass[10pt]{letter} % 10pt font size default, 11pt and 12pt are also possible

\usepackage{geometry} % Required for adjusting page dimensions

%\longindentation=0pt % Un-commenting this line will push the closing "Sincerely," to the left of the page

\geometry{
	paper=a4paper, % Change to letterpaper for US letter
	top=2cm, % Top margin
	bottom=1.5cm, % Bottom margin
	left=2.5cm, % Left margin
	right=2.5cm, % Right margin
	%showframe, % Uncomment to show how the type block is set on the page
}

\usepackage[T1]{fontenc} % Output font encoding for international characters
\usepackage[utf8]{inputenc} % Required for inputting international characters
\usepackage{color}
\usepackage{stix} % Use the Stix font by default

\usepackage{microtype} % Improve justification
\usepackage{dashrule}

%----------------------------------------------------------------------------------------
%	YOUR NAME & ADDRESS SECTION
%----------------------------------------------------------------------------------------

%\signature{John Smith} % Your name for the signature at the bottom

%\address{123 Broadway \\ City, State 12345 \\ (000) 111-1111} % Your address and phone number

%----------------------------------------------------------------------------------------
\newcommand{\mdf}[1]{\textcolor[rgb]{1.00,0.00,1.00}{#1}}
\begin{document}
	\textbf{Dear Editor,}\\
	Thank you for handling our manuscript. Our submission paper1027 entitled ``Preventing Self-intersection with Cycle Regularization in Mesh
	Reconstruction Networks'' is reviewed in GMP2019 and recommended to your journal for a fast-track submission process. We find the reviewers’ comments to be very valuable and helpful in improving our presentation, as well as important for guiding significantly to our research. We have read the comments carefully and the manuscript have been rechecked. Our modifications are provided in the revised manuscript according to these comments and suggestions. All the modifications are \mdf{highlighted in magenta} in the revised manuscript by the macro $\backslash$mdf in the source file. Listed below are our point-by-point responses to the reviewer comments.\\
	With all best wishes,\\
	Sincerely,
	
	\hdashrule{\linewidth}{1pt}{1mm}
	Chair:
	
	Q: \emph{The revision should focus on the case of models of non-zero genus.}
	
	A: As a response to chair's directives, we have added extra experiments in Sec 4.1 of our revised paper to try our proposed cycle regularization on case of models of genus-1. We discussed the effect of our cycle regularization in such cases. It is also our opinion that how to enable the neural network to learn to predict surface with arbitrary genus remains a challenging problem and is beyond the scope of this paper. 
	
	\hdashrule{\linewidth}{1pt}{1mm}
	%----------------------------------------------------------------------------------------
	Reviewer 21833:\\
	Q: \emph{I recommend to state the 2D image input directly in the title.}
	
	A:We have changed our title to ``Preventing Self-intersection with Cycle Regularization in Single RGB Image Based Mesh Reconstruction Networks''
	
	Q: \emph{The paper should also better distinguish between local an global self-intersections.}
	
	A:We have rephrased our discussion about self-intersections in introduction to better distinguish the local and global interference.
	
	Q: \emph{The structure of the paper is quite unfortunate. For example Section 3.1. refers to Definition 1 that appears later; this is very reader-unfriendly.}
	
	A:We have adjusted our manuscript to avoid such case as much as we can. However, we also want to avoid overcrowding of the figures.
	
	Q:\emph{Moreover, the paper speaks about injectivity of $f$ and self-overlapped points like these were two different phenomena.}
	
	A:Injectivity is a property of the regressed function $f$. Self-overlapped 3D points are observed defects on surface meshes. They maybe naturally related but we should still distinguish these two concepts.
	
	Q:\emph{Another rather confusing part is Def. 1 itself. Why do you formulate (1) and (2)? These are obviously equivalent and there is no reason to state them both.}
	
	A:We only keep statement (2) in our revised paper for your convenience.
	
	Q:\emph{Fig.8 is very small and the zoom-in is barely visible.  Some other figures should be provided in higher resolution, e.g. the *Input* row in Fig.6 is really bad.} 
	
	A:For input images, the images from the benchmark dataset are all low resolution images(224x224). This is now the standard setting with neural networks. It is difficult for us to provide high resolution version of them. For output meshes, we have adjusted the images in our revised paper for better view.
	
	Q:\emph{Looking at fig. 4, One wonders how many iterations are needed to receive results with desired chamfer distance. How do you set the number of iterations in your algorithm; what is the termination criterion?}
	
	A:We set the maximum iteration number to 1024. We have added such details in Sec 4.1 to elaborate on the experiment. we also add the curve of chamfer distance in Fig.4. 
	
	Q:\emph{What about objects with other genus than zero?}
	
	A:Solving this problem requires enabling neural network to learn to predict surfaces of arbitrary genus, which remains a challenge and beyond the scope of this paper. However, we do small exploration into the cases of genus-1 by manually selecting torus as source surface in our revised paper. Please see our revised Sec 4.1 for more details.
	
	\hdashrule{\linewidth}{1pt}{1mm}
	%----------------------------------------------------------------------------------------
	Reviewer 21924:\\
	
	We have added your recommendation of references as related work in our revised manuscript. Lifiting the limitation you have pointed out is beyond our reach for now. We will keep working on the problem.
	  
	
	\hdashrule{\linewidth}{1pt}{1mm}
	%----------------------------------------------------------------------------------------
	Reviewer 21930:\\
	
	Q: \emph{In Proposition 1, the free variable x should be bound by a quantifier.
		\\The first sentence of section 3.3 is describing a version of the Universal Approximation Theorem and would do well to refer to this.}
	
	A: We have fixed these two issues in our revised paper.
	
	Q:\emph{ A rather stringent limitation of the method is that it only applies to models with zero genus.}
	
	A: Lifting such limitation requires enabling neural network to learn to predict surfaces of arbitrary genus, which is still beyond our reach for now. However, we add some small exploration and discussion into the cases of genus-1. Please see Sec 4.1 and Figure~6 for details in our new manuscript. 

	
\end{document}