\section{Related Work}
3D reconstruction and modeling from a single image \mdf{have} been extensively studied as the problem of \emph{shape from monocular cues}, including shadings~\cite{shapefromshadingsurvey}, focuses~\cite{shapefromdf1,shapefromdf2}, and textures~\cite{Aloimonos1988}. 
These methods usually recover 2.5D surfaces from 2D images. 
Learning-based approaches, especially deep learning methods, can acquire more complicated priors by learning from datasets and recover much more complete 3D shape from a single image.
 
\subsection{General Learning Approaches}
As far as we known, early learning-based approaches can be traced back to \cite{Hoiem2007} and \cite{learn3D2007}. These methods learn to segment and classify regions in an image and finally produce a coarse 3D scene by folding the 2D image.
%
More recent techniques devide the problem to two stages \cite{Su:2014,jointimgshape}. They first retrieve shape components from a large dataset, and then assemble the components and deform the assembled shape to fit the observed image. These methods need to segment all object models into components for the database.
%
However, shape retrieval from images itself is an challenging problem due to the loss of information during 3D-to-2D projection. 
\cite{imgrecon15} avoid the retrieval step by learning a deformable 3D shape for each category and learn to predict deformation from an input image for these specific categories.
%
%These learning approaches are relatively early.
%A more ideal solution would be to directly learn 3D shapes from single images under an end-to-end framework.
\subsection{3D Neural Networks}
Most recently, researchers have developed techniques to represent 3D shapes inside deep learning frameworks. Unlike images, 3D shapes are not canonical functions on well-organized grids.
This leads to exploration on various representations of 3D shapes.

\noindent\textbf{Volume Occupancy}. 
An intuitive way to apply convolutional layers in 3D is to use volume occupancy of regular 3D grids to represent 3D shapes~\cite{3dshapenet}. It is subsequently used for 3D shape generation~\cite{3DR2N2,learnobj}.
%
The main disadvantage of volumetric representation was its large memory consumption due to the raising of dimension when extending 2D grids to 3D. 
Octree representation is proposed to support higher resolution outputs with limited memory, and used for shape generation~\cite{octreegen} and shape analysis~\cite{ocnn}.
%The most recent work of use an octree representation for shape generation (similar representation is used in for 3D shape classification and segmentation by \cite{ocnn}), which allows to higher resolution outputswith limited memory.

\noindent\textbf{Point Cloud}. 
Compared to regular 3D grids, point cloud representation is not limited by fixed local connections.
Many networks have been proposed to take unordered 3D point sets as input and extract geometric features from a 3D point set for classification or segmentation~\cite{pointnet,NIPS2017_7095,pointcnn}.
%
The first attempt to generate a set of discrete points from a single image using neural networks was made by \cite{PSGN}. However, it is non-trivial to construct continuous surface models from the predicted point sets, since the local variations of point positions are not continuous in the predicted point sets.

\noindent\textbf{Mesh}.
Meshes are widely used in game and movie industries.
In addition to vertex positions, mesh representation conveys local structures of vertices. 
However, mesh representation is not well supported in current neural networks.
% 
To generate mesh models using neural networks, composition weights of a series of base meshes are predicted by networks \cite{img2mesh} and \cite{endface}. %produce meshes by linear interpolating base meshes. 
Since it is only possible to choose or learn base meshes for a specific class of \mdf{objects}, these two networks only generate meshes for a specific class of objects, such as face.
%
In comparison, the idea of learning to map from a predefined domain as in AtlasNet  \cite{atlasnet} and Pixel2Mesh \cite{pixel2mesh}, can generate surface meshes for generic objects. Our work is a follow-up of their general idea and addresses a specific issue of surface self-intersection in their networks.
%but it .

\subsection{Parameterization for Neural Networks}
The idea of utilizing surface parameterization for neural networks has been explored by \cite{surfnet,geoimg}. 
Typically, a non-trainable procedure is involved for the creation of geometry image. 
Manifold surfaces are required as training data so that the shapes can be parameterized using spherical parameterization and turned into geometry images. However, public datasets like ShapeNet \cite{shapenetdata} contain meshes that are not manifold surfaces. 
In comparison, we are seeking techniques that can be integrated into networks and can be trained end-to-end along with the networks. Based on the same insight (self-intersection for surface mesh is related to non-injective mapping) as the parameterization techniques \mdf{(e.g.\cite{boundeddistortion,local-inj-map,provableplanarmapping,lifted_bijection,freeboundary,Liu_PP_2018})}, we propose a novel technique that is more suitable for training neural networks in an end-to-end manner. 

\subsection{Cycle Neural Networks}
The general idea of using neural networks to map from one domain to another domain and then map back has been utilized in previous works. In CycleGAN \cite{CycleGAN2017}, a famous example of such work, the cycle relation provides an extra constraint for translating between unpaired data from different domains. In comparison, our work uses the same general idea to enforce injectivity for 3D surface mesh generation networks and trying to prevent self-intersection in generated meshes. 

\subsection{Self-Intersection Removal}
Classic methods \mdf{(e.g. \cite{removeoffset,detectandremove,edgeswap,imofinter})} for removing self-intersections follow the pipeline that first identifies the self-intersected faces and then alters the faces with their proposed methods. However, it is difficult to integrate such a pipeline into nueral networks or to formulate their alteration operations in a differentiable manner. 

