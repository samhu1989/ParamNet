\section{Introduction}
%introduction to 3D shape reconstruction from single view
Inferring 3D shape from a single view image is a traditional problem for computer vision. In computer graphics, 3D modeling with a given image has also been extensively studied. \comments{Though it have been studied for decades, the problem remains challenging, due to the fact that 3D-to-2D projection is not invertible, and large portions of the 3D shape features are excluded in the 2D image.} In recent years, great success has been achieved for 3D shape reconstruction from a single color image using deep neural networks~\cite{3DR2N2,PSGN} \todo{add more refs}. Unlike classic shape from X \todo{add refs} approaches, these neural networks are able to recover not only the visible frontal shape but also the invisible part for object from a single view image by learning complicate prior knowledge from dataset. 

These networks all rely on variants of convolution neural network to extract information and encode 2D images, but use quite different techniques to represent and decode 3D shapes. Started by 3D ShapeNets\cite{3dshapenet} and greatly improved by introducing octree structure\cite{octreegen}, volumetric representation and 3D convolution networks are most commonly used in this problem. There is also point set generation network\cite{PSGN} that use unordered point set representation and directly regress point set using both convolution and fully connected breaches. Other interesting approaches includes \cite{endface} which use bilinear model to represent shape of face and regress the interpolation coefficient to generate the shape and \cite{surfnet} which explicitly employ spherical parameterization as post-processing to represent shape as geometry image in parameter domain and so on.

%introduction to 3d mesh reconstruction networks-- AtlasNet and Pixel2Mesh to be specific
In latest works, AtlasNet\cite{atlasnet} and Pixel2Mesh\cite{pixel2mesh} to be specific, a new idea that let the network learns to map or deform from a predefined surface (square and sphere for AtlasNet and ellipsoid for Pixel2Mesh) instead of directly regress the absolute position of surface point have been applied on this problem.
%introduction to the self-intersection issue

%reveal the cause of this problem (the disadvantage of chamferdistance loss)

%our proposed solution 

\comments{


By using convolutional layers on regular grids or multi-layer
perception on unordered 3D coordinates, the estimated 3D shape is represented
as either a volume occupancy~\cite{3DR2N2} or point cloud~\cite{PSGN} in neural networks. 
However, both representations lose important surface details, and they do not recover continuous surface forms.

While meshes are more suitable for modeling shape details, deformation and rendering with various materials, efforts have been made to produce mesh models using deep neural networks. 
Integrating mesh morphing techniques, an end-to-end trainable network is proposed to generate 3D meshes of a specific class of objects, such as faces~\cite{endface}.
However, the use of morphing technique limits the network's generalization ability on general objects.
%

In order to develop an end-to-end trainable network that is capable of producing meshes for objects from generic categories, we have proposed a framework of \emph{parameterization network} in our previous unpublished work. Instead of predicting the coordinates of vertices in the 3D ambient space, \emph{parameterization network} learn to predict the parameterization map of a surface.
Such framework have following advantages:

a) It is challenging to obtain a canonicalized triangulation of arbitrary surfaces, letting alone making it differential. With parameterization network, the triangulation can be first conducted inside a predefined convex domain (e.g. a rectangle patch, a sphere surface) and then transferred to the target surface.

b) With triangulation established, \emph{parameterization network} makes it convenient to integrate some useful classical mesh operations into neural network.

c) By designing the network to generate intermediate output in 3D space, \emph{parameterization network} can easily obtain more interpretable intermediate output and showing how the network progressively map the points from parameter domain to target shape.   

Some new works, AtlasNet\cite{atlasnet} and FoldingNet\cite{foldingnet} , sharing same concept with us, have been published before ours. AtlasNet\cite{atlasnet} have provided some useful theoretic guarantee for \emph{parameterization network} and clearly demonstrated the advantage a) in their results. FoldingNet\cite{foldingnet}, though does not explicitly address the problem of mesh reconstruction from single image, essentially used the same concept and by its network design, have clearly demonstrated the advantage c). 

With these two works before ours, in this paper, we re-base our implementations onto these two new works and show our broader exploration into configurations of the surface decoder inside \emph{parameterization network}.

A surface decoder, as the key part in \emph{parameterization network} (including AtlasNet\cite{atlasnet} and FoldingNet\cite{foldingnet}), accomplishes two things to involve the feature from image (for the task to generate shape from single image) and also map from the parameter domain to the target surface:
 
a) association: it associates the image representation with each point sampled from the parameter domain. 
 
b) point-wise decoding: it decodes the associated representations into 3D points in a point-wise manner.
 
 AtlasNet\cite{atlasnet} and FoldingNet\cite{foldingnet} both have chosen \emph{duplicate + concatenate} as techniques for association and they both have chosen MLP (multi-layer perceptions) as the building block for point-wise decoding.
These are intuitive choices. In this paper, we will explore and evaluate more possible techniques for association and more building blocks for point-wise decoding.

In addition, we want to demonstrate the forementioned advantage b) that has been left out by AtlasNet\cite{atlasnet} and FoldingNet\cite{foldingnet}. We use Laplace smooth as a layer in our network. We take it as a simple but representative example showing that how the \emph{parameterization network} makes it convenient to integrate classic mesh operation into neural network. 
}


 