\section{Method}
In this section, we firstly show that the self-intersection and self-overlap appear because the mapping is non-injective. 
Then we propose the cycle regularization technique and explain in details about how we respectively apply this general technique onto AtlasNet\cite{atlasnet} and Pixel2Mesh\cite{pixel2mesh}, whose network structures are different from each other. 
\subsection{Injective mapping for mesh generation}
\label{subsec:inj}
\begin{m_def}
Let $f$ be a function whose domain is a set $X$. The function $f$ is said to be injective provided that
\begin{equation}
\forall a,b \in X, f(a) = f(b) \Rightarrow a = b.
\end{equation}
Equivalently, 
\begin{equation}
\forall a,b \in X, a \neq b \Rightarrow f(a) \neq f(b).
\end{equation}
\end{m_def}
\subsection{Cycle regularization}
\label{subsec:cylcereg}
\begin{m_thm}
functions with left inverses are always injections. That is, given $f:X \rightarrow Y$, if there is a function $g:Y \rightarrow X$ such that,
\begin{equation}
\forall x \in X,~g(f(x)) = x,
\end{equation}
then $f$ is injective.
\end{m_thm}