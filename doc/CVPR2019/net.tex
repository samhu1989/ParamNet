\section{Techniques and building blocks}
In this section, we will elaborate on the techniques and building blocks that we have explored for the surface decoder. 
\subsection{Association techniques}
Association techniques are used to associate the image representation $\vec{x}$ with each point $\vec{z}_{n={0,1,\dots,N}}$ from $\mathbf{Z}=[\vec{z_0},\vec{z_1},\dots,\vec{z_N}]$ that are 2D or 3D coordinates.

\noindent\textbf{duplicate + concatenate}
this is most intuitive choice of and have been used for 
\begin{equation}
ass(\vec{x},\mathbf{Z})=\left[
\begin{aligned}
~&\vec{z_0},&\vec{z_1},\dots,&\vec{z_N}\\
~&\vec{x}  ,&\vec{x}~,\dots,&\vec{x}
\end{aligned}
\right].
\end{equation}
\noindent\textbf{duplicate + outer-product} 
Inspired by the success of bilinear pooling, we use duplicate + outer-product as a new association technique for surface decoder. 
\begin{equation}
ass(\vec{x},\mathbf{Z})=\left[
~\\vec(\vec{z_0}\vec{x}^T),\\vec(\vec{z_1}\vec{x}^T),\dots,\\vec(\vec{z_N}\vec{x}^T)\\
\right],
\end{equation}
in which $\\vec(\mathbf{X})$ means vectorization of matrix $\mathbf{X}$ by concatenating its columns.

\noindent\textbf{\emph{K}-neighbor point convolution}
\begin{equation}
ass(\vec{x},\mathbf{Z})=
\end{equation}

\subsection{Blocks for point-wise decoding}

\noindent\textbf{MLP}

\noindent\textbf{\emph{K}-neighbor point convolution}

\subsection{Laplace smooth layer}

\section{Network structures}
