\section{Method}
In this section, we firstly show that the we can make sure that there is no overlapped points on target surface by enforcing the mapping to be injective. 
Then we propose the cycle regularization technique and explain in details about how we respectively apply this general technique onto AtlasNet\cite{atlasnet} and Pixel2Mesh\cite{pixel2mesh}, whose network structures are different from each other. 
\subsection{Injective mapping and self-overlapped points}
\label{subsec:inj}
Start with the definition of injective mapping at \textbf{Definition}~\ref{def:injective}, we can intuitively induce the conclusion that given a predefined surface with $ X =\{\mathbf{x}~|~\mathbf{x}$ is a point on the predefined surface $ \} $, a target surface with $ Y =\{\mathbf{y}~|~\mathbf{y}$ is a point on the target surface $ \} $ and a function $f:X \rightarrow Y$. If $\exists$ $ \mathbf{a},\mathbf{b} \in X$, $\mathbf{a} \neq \mathbf{b}$ and $f(\mathbf{a}) = f(\mathbf{b})$ (i.e. the overlapped points on target surface exists) then by definition, $f$ is not an injective function. Equivalently (as the converse negative proposition), we can make sure there is no self-overlapped points on the target surface by enforcing $f$ to be injective.
\todo{add figure to show relation between injective mapping and overlapped points}
\begin{m_def}
\label{def:injective}
Let $f$ be a function whose domain is a set $X$. The function $f$ is said to be injective provided that
\begin{equation}
\forall a,b \in X, f(a) = f(b) \Rightarrow a = b.
\end{equation}
Equivalently, 
\begin{equation}
\forall a,b \in X, a \neq b \Rightarrow f(a) \neq f(b).
\end{equation}
\end{m_def}

\subsection{Cycle regularization}
\label{subsec:cyclereg}
\begin{m_thm}
\label{thm:injective}
functions with left inverses are always injective. That is, given $f:X \rightarrow Y$, if there is a function $g:Y \rightarrow X$ such that,
\begin{equation}
\forall x \in X,~g(f(x)) = x,
\end{equation}
then $f$ is injective.
\end{m_thm}

Based on \textbf{Theorem}~\ref{thm:injective}, we relax an injective constraint to our cycle regularization term as:
\begin{equation}
cycle_X(f)=\min_g\sum_{\mathbf{x}\in X}||g(f(\mathbf{x})) - \mathbf{x}||_2^2.
\end{equation}
By minimizing this term to zero:
\begin{equation}
f^* = \arg\min_f cycle_X(f)
\end{equation}
we can get the $f^*$ that has the left inverse function $g$, therefore $f^*$ is injective. 

However, it is never possible to actually minimize a regularization term to zero, especially in training neural networks. But when this term is minimized to sufficiently small then we can construct $g^*$ based on $g$ that is the left inverse function of $f^*$ and guarantee that $f^*$ is injective. A very tight example is that when the regularization term is so small that $g(f(\mathbf{x}))$ is closer to $\mathbf{x}$ than any other point in $X$. Such example can be summarized by \textbf{Proposition}~\ref{prop:nearest}.

\begin{m_prop}
	\label{prop:nearest}
	Given finite sets $X$ and $Y$ that are subset of Euclidean space $\mathcal{R}^3$, function $f:X \rightarrow Y$  and function $g:Y \rightarrow X$, if
	\begin{equation}
	\exists~g, \max_{\mathbf{x}\in X}|| g(f(\mathbf{x})) - \mathbf{x} ||_2^2 < \min_{\mathbf{a},\mathbf{b} \in X}|| \mathbf{a} - \mathbf{b} ||_2^2,
	\end{equation}
	then $f$ is injective.
\end{m_prop}

\textbf{Proposition}~\ref{prop:nearest} can be proved by simplely composite nearest neighbor with the function $g$. We can construct nearest neighbor function $l: \mathcal{R}^3 \rightarrow X $ as:
\begin{equation}
\forall \mathbf{a} \in \mathcal{R}^3, l(\mathbf{a}) = \arg\min_{\mathbf{b} \in X} || \mathbf{a} - \mathbf{b} ||_2^2
\end{equation}
then
\begin{equation}
\begin{aligned}
&\max_{\mathbf{x}\in X}|| g(f(\mathbf{x})) - \mathbf{x} ||_2^2 < \min_{\mathbf{a},\mathbf{b} \in X}|| \mathbf{a} - \mathbf{b} ||_2^2\\
&\Rightarrow || g(f(\mathbf{x})) - \mathbf{x} ||_2^2 \leq \min_{\mathbf{a},\mathbf{b} \in X}|| \mathbf{a} - \mathbf{b} ||_2^2\\
&\Rightarrow || g(f(\mathbf{x})) - \mathbf{x} ||_2^2 \leq \min_{\mathbf{b} \in X}|| g(f(\mathbf{x})) - \mathbf{b} ||_2^2\\
&\Rightarrow l(g(f(\mathbf{x}))) = \mathbf{x}\\
\end{aligned}
\end{equation}
then $g^*(\mathbf{y}) = l(g(\mathbf{y}))$ is the left inverse of $f$, therefore $f$ is injective. On mesh representation, such injective function $f$ for finite set can be further extend to entire continue surface by locally linear interpolation on each triangle.
\subsection{Implementation along with networks}
As stated in AtlasNet\cite{atlasnet}, it is possible to use multilayer perceptron with ReLU nonlinearities and enough hidden units to approximate any shape within a small positive error $\epsilon$. In practice, we employ another 3D surface decoder to approximate $g$. Then we explain how we implement this technique for AtlasNet and Pixel2Mesh respectively. Generally speaking, we reuse the network structures from their network respectively and show that our cycle regularization is a general technique for this type of networks. 

\noindent\textbf{AtlasNet} Depending on a shape representing feature $\mathbf{s}$, the AtlasNet use point-wise MLP
$f$ with parameters $\theta_f$ to learn to map points in $X=\{\mathbf{x}| \mathbf{x}$ are points uniformally sampled from predefined surface $P\}$ to points in $Y=\{\mathbf{y}| \mathbf{y}$ are points uniformally sampled from surface $S\}$. In implementation, $P$ can be either unit square $[0,1]^2$ or the surface of a 3D sphere. $S$ is the target surface, which is usually the surface of an object in AtlasNet\cite{atlasnet} and in this paper. Then we use another point-wise MLP $g$ with parameters $\theta_g$ to map points from $Y$ back to $X$. Along with our cycle regularization  total function can be written as:
\begin{equation}
\begin{aligned}
\mathcal{L}_{(X,Y)}(\theta_f) &= \sum_{\mathbf{x} \in X} \min_{\mathbf{y} \in Y}|| f_{\theta_f}(\mathbf{x};\mathbf{s}) - \mathbf{y} ||_2^2 \\ &+ \sum_{ \mathbf{y} \in Y}\min_{ \mathbf{x} \in X} || f_{\theta_f}(\mathbf{x};\mathbf{s}) - \mathbf{y} ||_2^2 \\ &+ \min_{\theta_g}\sum_{\mathbf{x} \in X}||g_{\theta_g}(f_{\theta_f}(\mathbf{x};\mathbf{s});\mathbf{s}) - \mathbf{x}||_2^2,
\end{aligned}
 \end{equation}
In which, the shape representation feature is simply concatenated to each point so that the $f$ and $g$ is depending on a global shape representing feature $\mathbf{s}$. In AtlasNet, $\mathbf{s}$ is generated from either PointNet for auto-encoding or ResNet-18 for single view reconstruction.

\noindent\textbf{Pixel2Mesh}
Comparing to AtlasNet, the Pixel2Mesh use a more complicate network structure.
\todo{describe and then implement the cycle regularization with pixel2mesh}


